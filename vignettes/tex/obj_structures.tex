\chapter{Package MARSS:  Object structures}

\section*{Model objects: class marssMODEL}
Objects of class `marssMODEL'\index{objects!marssMODEL} specify Multivariate Autoregressive State Space (MARSS) models. The \verb@model@ component of an ML estimation object (class marssMLE; see below) is a marssMODEL object. These objects have the following components:

  \begin{description}
    \item[\textbf{data}]{ An optional matrix (not dataframe), in which each row is a time series (time across columns). }
    \item[\textbf{fixed}]{ A list with 8 matrices Z, A, R, B, U, Q, x0, V0, specifying which elements of each parameter are fixed. }
    \item[\textbf{free}]{ A list with 8 matrices Z, A, R, B, U, Q, x0, V0, specifying which elements of each parameter are to be estimated. }
    \item[\textbf{M}]{ An array of dim $n \times n \times T$ (an $n \times n$ missing values matrix for each time point).  Each matrix is diagonal with 0 at the $i,i$ value if the $i$-th value of $\yy$ is missing, and 1 otherwise.} 
    \item[\textbf{miss.value}]{ Deprecated. Replace missing values with NAs before passing to MARSS. }
  \end{description}

  The matrices in \verb@fixed@ and \verb@free@ work as pairs to specify the fixed and free elements for each model parameter. The dimensions for \verb@fixed@ and \verb@free@ matrices are as follows, where $n$ is the number of observation time series and $m$ is the number of state processes:
  \begin{description}
    \item[\textbf{Z}]{ n x m }
    \item[\textbf{B}]{ m x m }
    \item[\textbf{U}]{ m x 1 }
    \item[\textbf{Q}]{ m x m }
    \item[\textbf{A}]{ n x 1 }
    \item[\textbf{R}]{ n x n }
    \item[\textbf{x0}]{ m x 1 }
    \item[\textbf{V0}]{ m x m }
  \end{description} 

Use \verb@is.marssMODEL()@\index{functions!is.marssMODEL} to check whether an marssMODEL object is correctly specified. The MARSS package includes an \verb@as.marssMODEL()@\index{functions!as.marssMODEL} method to convert objects of class popWrap (see next section) to objects of class marssMODEL. 

\section*{MARSSinputs}\index{objects!popWrap}

All the user inputs to a MARSS() call are put into a list and then passed to a function called MARSS.form() (where form is the text specified by the form argument) which used the model argument to creates the marssMODEL object and then MARSScheckinputs() to error in the others arguments.

  \begin{description}
  
  \item[\textbf{data}]{ A matrix (not dataframe) of observations (rows) $\times$ time (columns).  }
  \item[\textbf{model}]{ The specification is form dependent.  For the default marxss form, the inputs are a list with up to 14  elements Z, A, R, B, U, Q, x0, V0, C, c, D, d, tinitx, diffuse }
  \item[\textbf{inits}]{ A list with up to 10 matrices Z, A, R, B, U, Q, x0, V0, C, D specifying initial values for parameters. Dimensions are given in the class `marssMODEL' section. }
  \item[\textbf{miss.value}]{ Deprecated. Specifies missing value representation (default is NA). }
  \item[\textbf{method}]{ The method used for estimation: `kem' for EM, `BFGS' for quasi-Newton.}
  \item[\textbf{form}]{ The form to use to interpret the `model' argument and creat the marssMODEL object.}
  \item[\textbf{control}]{ List of estimation options. These are method dependent. }  
  \end{description}

Component \verb@model@ is a convenient way to specify model structure for certain common cases.  If \verb@model="use fixed/free"@, both \verb@fixed@ and \verb@free@ must be provided.  See the class marssMODEL section for how to specify fixed and free matrices.  The function \verb@MARSS()@\index{functions!MARSS} calls \verb@popWrap()@ to create a popWrap object, then \verb@is.marssMODEL()@ to coerce this object to class marssMODEL for the estimation function.

  The \verb@popWrap()@\index{functions!popWrap} function calls \verb@checkPopWrap()@\index{functions!checkPopWrap} to check user inputs from \verb@MARSS()@. Valid model structures are below.  

  \begin{description}   
    \item[\textbf{A}]{ String `unconstrained'=`unequal', `equal', `scaling' or `zero'. May also be a $m \times 1$ numeric matrix specifying a fixed $\aa$ matrix. Specified as a list matrix to allow fixed and estimated elements. }
    \item[\textbf{B}]{ String `identity', `zero', `unconstrained', `diagonal and unequal', `diagonal and equal', or `equalvarcov'. May also be a numeric matrix specifying a fixed $\BB$ matrix or a list matrix with fixed and estimated elements. } 
    \item[\textbf{Q}]{ String `identity', `zero', `unconstrained', `diagonal and unequal', `diagonal and equal', or `equalvarcov'. May also be a numeric matrix specifying a fixed $\QQ$ matrix or a list matrix with fixed and estimated elements. } 
    \item[\textbf{R}]{ String `identity', `zero', `unconstrained', `diagonal and unequal', `diagonal and equal', or `equalvarcov'. May also be a numeric matrix specifying a fixed $\RR$ matrix or a list matrix with fixed and estimated elements. }
    \item[\textbf{V0}]{ String `identity', `zero', `unconstrained', `diagonal and unequal', `diagonal and equal', or `equalvarcov'. May also be a numeric matrix specifying a fixed $\LAM$ matrix or a list matrix with fixed and estimated elements. }
    \item[\textbf{U}]{ String `unconstrained'=`unequal', `equal', or `zero'. May also be a $m \times 1$ numeric matrix specifying a fixed $\uu$ matrix. Specified as a list matrix to allow fixed and estimated elements.}
    \item[\textbf{x0}]{ String `unconstrained'=`unequal', `equal', or `zero'. May also be a $m \times 1$ numeric matrix specifying a fixed $\pipi$. Specified as a list matrix to allow fixed and estimated elements.}
    \item[\textbf{Z}]{ A vector  of class factor specifying which $\yy$ time series correspond to which state time series (in $\xx$) or a numeric $n \times m$ matrix specifying the $\ZZ$ matrix. The string `identity' can be used to specify a $n \times n$ identity matrix and string `ones' may be used to specify a column vector of $n$ ones.  A list matrix is used to specify a combination of fixed and estimated elements.}
  \end{description}
  

\section*{ML estimation objects: class marssMLE}\index{objects!marssMLE}

Objects of class marssMLE\index{objects!marssMLE} specify maximum-likelihood estimation for a MARSS model, both before and after fitting. A minimal marssMLE object contains components \verb@model, start@ and \verb@control@, which must be present for estimation by functions like \verb@MARSSkem()@\index{functions!MARSSkem}.

  \begin{description}
    \item[\textbf{model}]{ MARSS model specification (an object of class `marssMODEL'). }
    \item[\textbf{start}]{ List with 7 matrices A, R, B, U, Q, x0, V0, specifying initial values for parameters. Dimensions are given in the class marssMODEL section. }
    \item[\textbf{control}]{ A list specifying estimation options. For \verb@method="kem"@, these are
    \begin{description}
      \item[\textit{minit}]{ Minimum number of iterations in the maximization algorithm. } 
      \item[\textit{maxit}]{ Maximum number of iterations in the maximization algorithm. } 
      \item[\textit{abstol}]{ Optional tolerance for log-likelihood change.  If log-likelihood decreases less than this amount relative to the previous iteration, the EM algorithm exits. } 
      \item[\textit{trace}]{ A positive integer. If not zero, a record will be created of each variable the maximization iterations. The information recorded depends on the maximization method.}
      \item[\textit{safe}]{If TRUE, \verb@MARSSkem()@ will rerun \verb@MARSSkf()@ after each individual parameter update rather than only after all parameters are updated.  }
      \item{\textit{silent}}{ Suppresses printing of progress bar and convergence information. }    
    \end{description}
  }
  \end{description}

\verb@MARSSkem()@\index{functions!MARSSkem} appends the following components to the marssMLE' object: 

  \begin{description}
  \item[\textbf{method}]{ A string specifying the estimation method (`kem' for estimation by \verb@MARSSkem()@). }
  \item[\textbf{par}]{ A list with 8 matrices of estimated parameter values Z, A, R, B, U, Q, x0, V0. If there are fixed elements in the matrices, the corresponding elements in \verb@$par@ are set to the fixed values.}
  \item[\textbf{kf}]{ A list containing Kalman filter/smoother output. See Chapter \ref{chap:mainfunctions} }
  \item[\textbf{numIter}]{ Number of iterations required for convergence. }
  \item[\textbf{convergence}]{ Convergence status. }
  \item[\textbf{logLik}]{ the exact Log-likelihood. See Section \ref{sec:exactlikelihood}.}
  \item[\textbf{errors}]{ any error messages }
  \item[\textbf{iter.record}]{ record of the parameter values at each iteration (if \verb@control$trace=1@) }
\end{description}

Several functions append additional components to the `marssMLE' object\index{objects!marssMLE} passed in. These include:

  \begin{description}
  \item{\verb@MARSSaic@}{ Appends \verb@AIC, AICc, AICbb, AICbp@, depending on the AIC flavors requested.\index{functions!MARSSaic} }
  \item{\verb@MARSShessian@}{ Appends \verb@Hessian, gradient, parMean@ and \verb@parSigma@.\index{functions!MARSShessian} }
  \item{\verb@MARSSparamCIs@}{ Appends \verb@par.se, par.bias, par.upCI@ and \verb@par.lowCI@.\index{functions!MARSSparamCIs}}
  \end{description}

