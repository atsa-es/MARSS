\chapter{Textbooks and articles that use MARSS modeling for population modeling}
\label{chap:SSreferences}
\section*{Textbooks Describing the Estimation of Process and Non-process Variance}
There are many textbooks on Kalman filtering and estimation of state-space models.  The following are a sample of books on state-space modeling that we have found especially helpful.
\bigskip

Shumway, R. H., and D. S. Stoffer. 2006. Time series analysis and its applications. Springer-Verlag.

Harvey, A. C. 1989. Forecasting, structural time series models and the Kalman filter. Cambridge University Press.

Durbin, J., and S. J. Koopman. 2001. Time series analysis by state space methods. Oxford University Press.

Kim, C. J. and Nelson, C. R. 1999. State space models with regime switching. MIT Press.

King, R., G. Olivier, B. Morgan, and S. Brooks.  2009. Bayesian analysis for population ecology. CRC Press.

Giovanni, P., S. Petrone, and P. Campagnoli. 2009. Dynamic linear models in R. Springer-Verlag.

Pole, A., M. West, and J. Harrison. 1994. Applied Bayesian forecasting and time series analysis. Chapman and Hall.

Bolker, B. 2008. Ecological models and data in R.  Princeton University Press.

West, M. and Harrison, J. 1997. Bayesian forecasting and dynamic models. Springer-Verlag.

Tsay, R. S. 2010. Analysis of financial time series. Wiley.

\section*{Maximum-likelihood papers}
This is just a sample of the papers from the population modeling literature.
\bigskip

de Valpine, P. 2002. Review of methods for fitting time-series models with process and observation error and likelihood calculations for nonlinear, non-Gaussian state-space models. Bulletin of Marine Science 70:455-471.

de Valpine, P. and A. Hastings. 2002. Fitting population models incorporating process noise and observation error. Ecological Monographs 72:57-76.

de Valpine, P. 2003. Better inferences from population-dynamics experiments using Monte Carlo state-space likelihood methods. Ecology 84:3064-3077.

de Valpine, P. and R. Hilborn. 2005. State-space likelihoods for nonlinear fisheries time series. Canadian Journal of Fisheries and Aquatic Sciences 62:1937-1952.

Dennis, B., J.M. Ponciano, S.R. Lele, M.L. Taper, and D.F. Staples. 2006. Estimating density dependence, process noise, and observation error. Ecological Monographs 76:323-341.

Ellner, S.P. and E.E. Holmes. 2008. Resolving the debate on when extinction risk is predictable. Ecology Letters 11:E1-E5.

Erzini, K. 2005. Trends in NE Atlantic landings (southern Portugal): identifying the relative importance of fisheries and environmental variables.  Fisheries Oceanography 14:195-209.

Erzini, K., Inejih, C. A. O., and K. A. Stobberup. 2005. An application of two techniques for the analysis of short, multivariate non-stationary time-series of Mauritanian trawl survey data ICES Journal of Marine Science 62:353-359.

Hinrichsen, R.A. and E.E. Holmes. 2009. Using multivariate state-space models to study spatial structure and dynamics. In Spatial Ecology (editors Robert Stephen Cantrell, Chris Cosner, Shigui Ruan). CRC/Chapman Hall.

Hinrichsen, R.A. 2009. Population viability analysis for several populations using multivariate state-space models. Ecological Modelling 220:1197-1202.

Holmes, E.E. 2001. Estimating risks in declining populations with poor data. Proceedings of the National Academy of Sciences of the United States of America 98:5072-5077.

Holmes, E.E. and W.F. Fagan. 2002. Validating population viability analysis for corrupted data sets. Ecology 83:2379-2386.

Holmes, E.E. 2004. Beyond theory to application and evaluation: diffusion approximations for population viability analysis. Ecological Applications 14:1272-1293.

Holmes, E.E., W.F. Fagan, J.J. Rango, A. Folarin, S.J.A., J.E. Lippe, and N.E. McIntyre. 2005. Cross validation of quasi-extinction risks from real time series: An examination of diffusion approximation methods. U.S. Department of Commerce, NOAA Tech. Memo. NMFS-NWFSC-67, Washington, DC.

Holmes, E.E., J.L. Sabo, S.V. Viscido, and W.F. Fagan. 2007. A statistical approach to quasi-extinction forecasting. Ecology Letters 10:1182-1198.

Kalman, R.E. 1960.  A new approach to linear filtering and prediction problems. Journal of Basic Engineering 82:35-45.

Lele, S.R. 2006. Sampling variability and estimates of density dependence: a composite likelihood approach.  Ecology 87:189-202.

Lele, S.R., B. Dennis, and F. Lutscher. 2007. Data cloning: easy maximum likelihood estimation for complex ecological models using Bayesian Markov chain Monte Carlo methods. Ecology Letters 10:551-563.

Lindley, S.T. 2003. Estimation of population growth and extinction parameters from noisy data. Ecological Applications 13:806-813.

Ponciano, J.M., M.L. Taper, B. Dennis, S.R. Lele. 2009. Hierarchical models in ecology: confidence intervals, hypothesis testing, and model selection using data cloning. Ecology 90:356-362.

Staples, D.F., M.L. Taper, and B. Dennis. 2004. Estimating population trend and process variation for PVA in the presence of sampling error. Ecology 85:923-929.

Zuur, A. F., and G. J. Pierce. 2004. Common trends in Northeast Atlantic squid time series. Journal of Sea Research 52:57-72.

Zuur, A. F., I. D. Tuck, and N. Bailey. 2003. Dynamic factor analysis to estimate common trends in fisheries time series. Canadian Journal of Fisheries and Aquatic Sciences 60:542-552.

Zuur, A. F., R. J. Fryer, I. T. Jolliffe, R. Dekker, and J. J. Beukema. 2003. Estimating common trends in multivariate time series using dynamic factor analysis. Environmetrics 14:665-685.

\section*{Bayesian papers}
This is a sample of the papers from the population modeling and animal tracking literature.
\bigskip

Buckland, S.T., K.B. Newman, L. Thomas and N.B. Koestersa. 2004. State-space models for the dynamics of wild animal populations. Ecological modeling 171:157-175.

Calder, C., M. Lavine, P. M{\"u}ller, J.S. Clark. 2003. Incorporating multiple sources of stochasticity into dynamic population models. Ecology 84:1395-1402.

Chaloupka, M. and G. Balazs.  2007.  Using Bayesian state-space modelling to assess the recovery and harvest potential of the Hawaiian green sea turtle stock. Ecological Modelling 205:93-109.

Clark, J.S. and O.N. Bj{\o}rnstad. 2004. Population time series: process variability, observation errors, missing values, lags, and hidden states. Ecology 85:3140-3150.

Jonsen, I.D., R.A. Myers, and J.M. Flemming. 2003. Meta-analysis of animal movement using state space models. Ecology 84:3055-3063.

Jonsen, I.D, J.M. Flemming, and R.A. Myers. 2005. Robust state-space modeling of animal movement data. Ecology 86:2874-2880.

Meyer, R. and R.B. Millar. 1999. BUGS in Bayesian stock assessments. Can. J. Fish. Aquat. Sci. 56:1078-1087.

Meyer, R. and R.B. Millar. 1999. Bayesian stock assessment using a state-space implementation of the delay difference model. Can. J. Fish. Aquat. Sci. 56:37-52.

Meyer, R. and R.B. Millar. 2000. Bayesian state-space modeling of age-structured data: fitting a model is just the beginning. Can. J. Fish. Aquat. Sci. 57:43-50.

Newman, K.B., S.T. Buckland, S.T. Lindley, L. Thomas, and C. Fern{\'a}ndez. 2006. Hidden process models for animal population dynamics. Ecological Applications 16:74-86.

Newman, K.B., C. Fern{\'a}ndez, L. Thomas, and S.T. Buckland. 2009.  Monte Carlo inference for state-space models of wild animal populations. Biometrics 65:572-583

Rivot, E., E. Pr{\'e}vost, E. Parent, and J.L. Baglini{\`e}re. 2004.  A Bayesian state-space modelling framework for fitting a salmon stage-structured population dynamic model to multiple time series of field data. Ecological Modeling 179:463-485.

Schnute, J.T. 1994. A general framework for developing sequential fisheries models. Canadian J. Fisheries and Aquatic Sciences 51:1676-1688.

Swain, D.P., I.D. Jonsen, J.E. Simon, and R.A. Myers. 2009. Assessing threats to species at risk using stage-structured state-space models: mortality trends in skate populations.  Ecological Applications 19:1347-1364.

Thogmartin, W.E., J.R. Sauer, and M.G. Knutson. 2004. A hierarchical spatial model of avian abundance with application to cerulean warblers. Ecological Applications 14:1766-1779.

Trenkel, V.M., D.A. Elston, and S.T. Buckland. 2000. Fitting population dynamics models to count and cull data using sequential importance sampling. J. Am. Stat. Assoc. 95:363-374.

Viljugrein, H., N.C. Stenseth, G.W. Smith, and G.H. Steinbakk. 2005. Density dependence in North American ducks. Ecology 86:245-254.

Ward, E.J., R. Hilborn, R.G. Towell, and L. Gerber. 2007. A state-space mixture approach for estimating catastrophic events in time series data. Can. J. Fish. Aquat. Sci., Can. J. Fish. Aquat. Sci. 644:899-910.

Wikle, C.K., L.M. Berliner, and N. Cressie. 1998. Hierarchical Bayesian space-time models. Journal of Environmental and Ecological Statistics 5:117-154

Wikle, C.K. 2003. Hierarchical Bayesian models for predicting the spread of ecological processes. Ecology 84:1382-1394.
