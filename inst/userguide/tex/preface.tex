\subsection*{Preface}
\addcontentsline{toc}{subsection}{Acknowledgments} 

The initial motivation for our work with MARSS models was a collaboration with Rich Hinrichsen. Rich developed a framework for analysis of multi-site population count data using MARSS models and bootstrap AICb \citep{HinrichsenHolmes2009}. Our work (EEH and EJW) extended Rich's framework, made it more general, and led to the development of a parametric bootstrap AICb\index{model selection!bootstrap AIC, AICbp} for MARSS models, which allows one to do model-selection using datasets with missing values \citep{Wardetal2010, HolmesWard2010}.  Later, we developed additional algorithms for simulation and confidence intervals.  Discussions with Mark Scheuerell led to an extensive revision of the EM algorithm and to the development of a general EM algorithm for constrained MARSS models \citep{Holmes2010}.  Discussions with Mark also led to a complete rewrite of the model specification so that the package could be used for MARSS models in general---rather than simply the form of MARSS model used in our applications.  Many collaborators have helped test the package; we thank especially Yasmin Lucero, Kevin See, and Brice Semmens.  Development of the code into a \R package would not have been possible without Kellie Wills, who wrote much of the original package code outside of the algorithm functions.  Finally, we thank the participants of our MARSS workshops and courses and the MARSS users who have contacted us regarding issues that were unclear in the manual, errors, or suggestions regarding new applications.  Discussions with these users have helped us improve the manual and go in new directions.

The application chapters were developed originally as part of workshops on analysis of multivariate time-series data given at the Ecological Society of America meetings since 2005 and taught by us along with Yasmin Lucero, Stephanie Hampton, and Brice Semmens.  The chapter on extinction estimation and trend estimation was initially developed by Brice Semmens and later extended by us for this user guide. The algorithm behind the TMU figure in Chapter \ref{chap:CSpva} was developed during a collaboration with Steve Ellner \citep{EllnerHolmes2008}.  Later we further developed the chapters as part of a course we teach on analysis of fisheries and environmental time-series data at the University of Washington.  You can find online versions of our time-series analysis course and an eBook from the course on our Applied Time Series Analysis website \url{http://atsa-es.github.io}.
 
The authors are federal research scientists; EEH and EJW are with NOAA Fisheries and MDS is with USGS (and University of Washington).  This work was conducted as part of our jobs with United States federal government agencies.   A CAMEO grant from the National Science Foundation and NOAA Fisheries provided the initial impetus for the development of the package as part of a research project with Stephanie Hampton, Lindsay Scheef, and Steven Katz on analysis of marine plankton time series.  During the initial stages of this work, EJW was supported on a post-doctoral fellowship from the National Research Council and MDS was partially supported by a PECASE award from the White House Office of Science and Technology Policy.

You are welcome to use the code and adapt it with full attribution.  You should use citation \citet{Holmesetal2012} for the \{MARSS\} package.  It may not be used in any commercial applications nor may it be copyrighted.  Use of the EM algorithm should cite \citet{Holmes2010}. Links to more code and publications on MARSS applications can be found by following the links at our academic websites:
\begin{itemize}
\item \url{http://faculty.washington.edu/eeholmes}
\item \url{http://faculty.washington.edu/scheuerl}
\item \url{http://faculty.washington.edu/warde}
\end{itemize}