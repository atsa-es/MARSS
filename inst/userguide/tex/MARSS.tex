\chapter{The \texttt{MARSS()} function}\label{chap:MARSS}

From the user perspective, the main package function is \verb@MARSS()@.  This fits a MARSS model (Equation \ref{eqn:marss}) to a matrix of data. The function call takes the form:
\begin{Schunk}
\begin{Sinput}
MARSS(data, model=list(), form="marxss")) 
\end{Sinput}
\end{Schunk}
The model argument is a list with names \verb@B@, \verb@U@, \verb@C@, \verb@c@, \verb@Q@, \verb@Z@, \verb@A@, \verb@D@, \verb@d@, \verb@R@, \verb@x0@, \verb@V0@.  Elements can be left off to use default values.  The form argument tells \verb@MARSS()@ how to use the model list elements.  The default is \verb@form="marxss"@ which is the model in Equation \ref{eqn:marss}.

The data must be passed in as a $n \times T$ matrix (time goes across columns) or a ts object or vector (which will be converted to a $n \times T$ matrix with time across the columns). A data matrix consisting of three time series ($n=3$) with six time steps might look like
\begin{equation*}
\yy = \left[ \begin{array}{ccccccc}
    1 & 2 & NA & NA & 3.2 & 8\\
    2 & 5 &  3 & NA & 5.1 & 5\\
    1 & NA & 2 & 2.2 & NA & 7\end{array} \right]
\end{equation*}
where NA denotes a missing value.

The argument \verb@model@ specifies the structure of the MARSS model.  It is a list\index{model specification!in MARSS}, where the list elements for each model parameter specify the form of that parameter.  

The most general way to specify model structure is to use a list matrix.  The list matrix allows one to combine fixed and estimated elements in one's parameter specification.  It allows a one-to-one correspondence between how you write the parameter matrix on paper and how you specify it in \R.  For example, let's say $\QQ$ and $\uu$ have the following forms in your model:
\begin{equation*}
\QQ=
\begin{bmatrix}
q&0&0\\
0&q&0\\
0&0&1
\end{bmatrix}
\text{ and }
\uu=
\begin{bmatrix}
0.05\\
u_1\\
u_2
\end{bmatrix}
\end{equation*}
So $\QQ$ is a diagonal matrix with the 3rd variance fixed at 1 and the 1st and 2nd estimated and equal.  The 1st element of $\uu$ is fixed, and the 2nd and 3rd are estimated and different. You can specify this using a list matrix:
 \begin{Schunk}
\begin{Sinput}
Q=matrix(list("q",0,0,0,"q",0,0,0,1),3,3)
U=matrix(list(0.05,"u1","u2"),3,1)
\end{Sinput}
\end{Schunk}
If you print out \verb@Q@ and \verb@U@, you will see they look exactly like $\QQ$ and $\uu$ written above.  MARSS will keep the fixed values fixed and estimate $q$, $u1$, and $u2$.

List matrices allow the most flexible model structures, but \verb@MARSS()@ also has text shortcuts for a number of common model structures.  Below, the possible ways to specify each model parameter are shown, using $m=3$ (the number of hidden state processes) and $n=3$ (number of observation time series).


\section{$\uu$, $\aa$ and $\pipi$ model structures}

$\uu$, $\aa$ and $\pipi$ are all row matrices and the options for specifying their structures are the same.  $\aa$ has one special option, \verb@"scaling"@ described below.  The allowable structures are shown using $\uu$ as an example.   Note that you should be careful about specifying shared structure in $\pipi$ because you need to make sure the structure in $\LAM$ matches.  For example, if you require that all the $\pipi$ values are shared (equal) then $\LAM$ cannot be a diagonal matrix since that would be saying that the $\pipi$ values are independent, which they are clearly not if you force them to be equal.

\begin{itemize}\itemsep5pt
\item[] \texttt{U=matrix(list(),m,1)}: This is the most general form and allows one to specify fixed and estimated elements in $\uu$.   Each character string in $\uu$ is the name of one of the $\uu$ elements to be estimated.  For example if   \texttt{U=matrix(list(0.01,"u","u"),3,1)}, then $\uu$ in the model has the following structure:
\begin{equation*}
 \left[ \begin{array}{c}
    0.01 \\
    u \\
    u \end{array} \right]
\end{equation*}

\item[] \texttt{U=matrix(c(),m,1)}, where the values in \verb@c()@ are all character strings:  each character string is the name of an element to be estimated.  For example if   \texttt{U=matrix(c("u1","u1","u2"),3,1)}, then $\uu$ in the model has the following structure:
\begin{equation*}
 \left[ \begin{array}{c}
    u_1 \\
    u_1 \\
    u_2 \end{array} \right]
\end{equation*}
with two values being estimated.  \texttt{U=matrix(list("u1","u1","u2"),3,1)} has the same effect.

\item[] \texttt{U="unequal"} or \texttt{U="unconstrained"}: Both of these stings indicate that each element of $\uu$ is estimated. If $m=3$, then $\uu$ would have the form:
\begin{equation*}
 \left[ \begin{array}{c}
    u_1\\
    u_2\\
    u_3 \end{array} \right]
\end{equation*}

\item[] \texttt{U="equal"}: There is only one value in $\uu$:
\begin{equation*}
 \left[ \begin{array}{c}
    u \\
    u \\
    u \end{array} \right]
\end{equation*}

\item[] \texttt{U=matrix(c(),m,1)}, where the values in \verb@c()@ all numerical values: $\uu$ is fixed and has no estimated values.  If   \texttt{U=matrix(c(0.01,1,-0.5),3,1)}, then $\uu$ in the model is:
\begin{equation*}
 \left[ \begin{array}{c}
    0.01 \\
    1 \\
    -0.5 \end{array} \right]
\end{equation*}
\texttt{U=matrix(list(0.01,1,-0.5),3,1)} would have the same effect.

\item[] \texttt{U="zero"}: $\uu$ is all zero:
\begin{equation*}
 \left[ \begin{array}{c}
    0 \\
    0 \\
    0 \end{array} \right]
\end{equation*}

The $\aa$ parameter has a special option, \verb@"scaling"@, which is the default behavior.  In this case, $\aa$ is treated like a scaling parameter.  If there is only one $\yy$ row associated with an $\xx$ row, then the corresponding $\aa$ element is 0.  If there are more than one $\yy$ rows associated with an $\xx$ row, then the first $\aa$ element is set to 0 and the others are estimated.  For example, say $m=2$ and $n=4$ and $\ZZ$ looks like the following:
\begin{equation*}
 \ZZ =
  \left[ \begin{array}{cc}
    1 & 0  \\
    1 & 0  \\
    1 & 0  \\
    0 & 1  \end{array} \right]
\end{equation*}
Then the 1st-3rd rows of $\yy$ are associated with the first row of $\xx$, and the 4th row of $\yy$ is associated with the last row of $\xx$.  Then if $\aa$ is specified as \verb@"scaling"@, $\aa$ has the following structure:
\begin{equation*}
 \left[ \begin{array}{c}
    0 \\
    a_1 \\
    a_2 \\
    0 \end{array} \right]
\end{equation*}

\end{itemize}

\section{$\QQ$, $\RR$, $\LAM$ model structures}

The possible $\QQ$, $\RR$, and $\LAM$ model structures are identical, except that $\RR$ is $n \times n$ while $\QQ$ and $\LAM$ are $m \times m$.  All types of structures can be specified using a list matrix, but there are also text shortcuts for specifying common structures.  The structures are shown using $\QQ$ as the example.

\begin{itemize}\itemsep5pt
\item[] \texttt{Q=matrix(list(),m,m)}: This is the most general way to specify the parameters and allows there to be fixed and estimated elements.   Each character string in the list matrix is the name of one of the $\QQ$ elements to be estimated, and each numerical value is a fixed value.  For example if \newline   \texttt{Q=matrix(list("s2a",0,0,0,"s2a",0,0,0,"s2b"),3,3)}, \newline
then $\QQ$  has the following structure:
\begin{equation*}
 \left[ \begin{array}{ccc}
    \sigma^2_a & 0 & 0\\
    0 & \sigma^2_a & 0 \\
    0 & 0 & \sigma^2_b \end{array} \right]
\end{equation*}
Note that \texttt{diag(c("s2a","s2a","s2b"))} will not have the desired effect of producing a matrix with numeric 0s on the off-diagonals.  It will have character 0s and MARSS will interpret ``0'' as the name of an element of $\QQ$ to be estimated.  Instead, the following two lines can be used:
\begin{Schunk}
\begin{Sinput}
Q=matrix(list(0),3,3)
diag(Q)=c("s2a","s2a","s2b")
\end{Sinput}
\end{Schunk}

\item[] \texttt{Q="diagonal and equal"}: There is only one process variance value in this case:
\begin{equation*}
 \left[ \begin{array}{ccc}
    \sigma^2 & 0 & 0\\
    0 & \sigma^2 & 0 \\
    0 & 0 & \sigma^2 \end{array} \right]
\end{equation*}

\item[] \texttt{Q="diagonal and unequal"}:  There are $m$ process variance values in this case:  
\begin{equation*}
 \left[ \begin{array}{ccc}
    \sigma^2_1 & 0 & 0\\
    0 & \sigma^2_2 & 0 \\
    0 & 0 & \sigma^2_3 \end{array} \right]
\end{equation*}

\item[] \texttt{Q="unconstrained"}: There are values on the diagonal and the off-diagonals of $\QQ$ and the variances and covariances are all different:  
\begin{equation*}
 \left[ \begin{array}{ccc}
    \sigma^2_1 & \sigma_{1,2} & \sigma_{1,3}\\
    \sigma_{1,2} & \sigma^2_2 & \sigma_{2,3} \\
    \sigma_{1,3} & \sigma_{2,3} & \sigma^2_3 \end{array} \right]
\end{equation*}
There are $m$ process variances and $(m^2-m)/2$ covariances in this case, so $(m^2+m)/2$ values to be estimated.  Note that variance-covariance matrices are never truly unconstrained since the upper and lower triangles of the matrix must be equal.

\item[] \texttt{Q="equalvarcov"}: There is one process variance and one covariance:
\begin{equation*}
 \left[ \begin{array}{ccc}
    \sigma^2 & \beta & \beta\\
    \beta & \sigma^2 & \beta \\
    \beta & \beta & \sigma^2 \end{array} \right]
\end{equation*}

\item[] \texttt{Q=matrix(c(), m, m)}, where all values in \verb@c()@ are character strings:  Each element in $\QQ$ is estimated and each character string is the name of a value to be estimated. Note if $m=1$, you still need to wrap its value in \verb@matrix()@ so that its class is matrix.  

\item[] \texttt{Q=matrix(c(), m, m)}, where all values in \verb@c()@ are numeric values:  Each element in $\QQ$ is fixed to the values in the matrix. 

\item[] \texttt{Q="identity"}: The $\QQ$ matrix is the identity matrix:
\begin{equation*}
 \left[ \begin{array}{ccc}
    1 & 0 & 0\\
    0 & 1 & 0 \\
    0 & 0 & 1 \end{array} \right]
\end{equation*}

\item[] \texttt{Q="zero"}: The $\QQ$ matrix is all zeros:
\begin{equation*}
 \left[ \begin{array}{ccc}
    0 & 0 & 0\\
    0 & 0 & 0 \\
    0 & 0 & 0 \end{array} \right]
\end{equation*}
\end{itemize}

Be careful when setting $\LAM$ model structures.  Mis-specifying the structure of $\LAM$ can have catastrophic, but difficult to discern, effects on your estimates\index{prior}.  See the comments on priors in Chapter \ref{chap:intro}.

\section{$\BB$ model structures}

Like the variance-covariance matrices ($\QQ$, $\RR$ and $\LAM$), $\BB$ can be specified with a list matrix to allow you to have both fixed and shared elements in the $\BB$ matrix.  Character matrices and matrices with fixed values operate the same way as for the variance-covariance matrices.  In addition, the same text shortcuts are available: ``unconstrained", ``identity", ``diagonal and equal", ``diagonal and unequal", ``equalvarcov", and ``zero".  A fixed $\BB$ can be specified with a numeric matrix, but  all eigenvalues must fall within the unit circle; meaning \texttt{all(abs(eigen(B)\$values)<=1)} must be true.  

\section{$\ZZ$ model}

Like $\BB$ and the variance-covariance matrices, $\ZZ$ can be specified with a list matrix to allow you to have both fixed and estimated elements in $\ZZ$. If $\ZZ$ is a square matrix, many of the same text shortcuts are available: ``diagonal and equal", ``diagonal and equal", and ``equalvarcov".  If $\ZZ$ is a design matrix\footnote{a matrix with only 0s and 1s and where the row sums are all equal to 1}, then a special shortcut is available using \texttt{factor()} which allows you to specify which $\yy$ rows are associated with which $\xx$ rows.  See Chapter \ref{chap:Examples} and the applications chapters for more examples.

\begin{itemize}\itemsep5pt
\item[] \texttt{Z=factor(c(1,1,1))}:  All $\yy$ time series are observing the same (and only) hidden state trajectory $x$ ($n=3$ and $m=1$):
\begin{equation*}
 \ZZ =
 \left[ \begin{array}{c}
    1  \\
    1  \\
    1  
    \end{array} \right]  
\end{equation*}

\item[] \texttt{Z=factor(c(1,2,3))}:  Each time series in $\yy$ corresponds to a different hidden state trajectory.  This is the default $\ZZ$ model and in this case $n=m$:
\begin{equation*}
 \ZZ =
  \left[ \begin{array}{ccc}
    1 & 0 & 0 \\
    0 & 1 & 0 \\
    0 & 0 & 1 \end{array} \right]
\end{equation*}

\item[] \texttt{Z=factor(c(1,1,2))}: The first two time series in $\yy$ corresponds to one hidden state trajectory and the third $\yy$ time series corresponds to a different hidden state trajectory.  Here $n=3$ and $m=2$:
\begin{equation*}
 \ZZ =
  \left[ \begin{array}{cc}
    1 & 0  \\
    1 & 0  \\
    0 & 1  \end{array} \right]
\end{equation*}
The $\ZZ$ model can be specified using either numeric or character factor levels.  \verb@c(1,1,2)@ is the same as \verb@c("north","north","south")@

\item[] \texttt{Z="identity"}: This is the default behavior.  This means $\ZZ$ is a $n \times n$ identity matrix and $m=n$.  If $n=3$, it is the same as \texttt{Z=factor(c(1,2,3))}.

\item[] \texttt{Z=matrix(c(), n, m)}, where the elements in \verb@c()@ are all strings:  Passing in a $n \times m$ character matrix, means that each character string is a value to be estimated.  Be careful that you are specifying an identifiable model when using this option.

\item[] \texttt{Z=matrix(c(), n, m)}, where the elements in \verb@c()@ are all numeric:  Passing in a $n \times m$ numeric matrix means that $\ZZ$ is fixed to the values in the matrix. The matrix must be numeric but it does not need to be a design matrix.

\item[] \texttt{Z=matrix(list(), n, m)}:  Passing in a $n \times m$ list matrix allows you to combine fixed and estimated values in the $\ZZ$ matrix.  Be careful that you are specifying an identifiable model.

\end{itemize}

\section{Default model structures}
 The defaults for the model arguments in \verb@form="marxss"@ are
\begin{itemize}
\item[] \verb@Z="identity"@ each $y$ in $\yy$ corresponds to one $x$ in $\xx$
\item[] \verb@B="identity"@ no interactions among the $x$'s in $\xx$
\item[] \verb@U="unequal"@ the $u$'s in $\uu$ are all different
\item[] \verb@Q="diagonal and unequal"@ process errors are independent but have different variances
\item[] \verb@R="diagonal and equal"@ the observations are i.i.d.
\item[] \verb@A="scaling"@ $\aa$ is a set of scaling factors
\item[] \verb@C="zero"@ and \verb@D="zero"@ no inputs.
\item[] \verb@c="zero"@ and \verb@d="zero"@ no inputs.
\item[] \verb@x0="unequal"@ all initial states are different
\item[] \verb@V0="zero"@ the initial condition on the states ($\xx_0$ or $\xx_1$) is fixed but unknown
\item[] \verb@tinitx=0@ the initial state refers to $t=0$ instead of $t=1$.
\end{itemize}
